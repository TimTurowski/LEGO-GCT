\subsection{Zielbestimmung}

Heutzutage günstig Lego-Sets zu erwerben kann auf mehreren Ebenen ein wohlüberlegtes Unterfangen sein. Welche der Lego-Händler bieten die Sets günstig an? Ist es günstiger sich die Einzelteile des Sets einzeln zu bestellen? Was ist wenn bestimmte Einzelteile nicht mehr verfügbar sind? \newline
Das Ziel des Projekts ist genau diese Fragestellungen, durch die Entwicklung eines Tools zur Ermittlung der günstigsten Anbieter, zu beantworten. Das Tool soll dabei auch in der Lage sein, abzuwägen, ob es günstiger ist, das Set oder Einzelteile zu kaufen und zu kombinieren.

\subsubsection{Musskriterien}
\Roman{1} . Interne Teileverwaltung durch das Pflegen einer Datenbank über die Einzelteile. \newline 
\Roman{2} . Die PDF-Bauanleitungen der Lego-Sets sollen mit OCR auslesbar sein. \newline
\Roman{3} . Ein Crawler bezieht die Preise der unterschiedlichen Händler. \newline
\Roman{4} . Ein Preisvergleich unter den Händlern, im Bezug auf Kauf eines Sets oder die jeweiligen Einzelteile, findet statt. \newline
\Roman{5} . Darstellung des Tools als graphische Oberfläche. (Startseite, Benuterverwaltung, Preisvergleich, Warenkorb) \newline
\Roman{6} . Es soll möglich sein Benutzeraccounts anzulegen und diese verwalten zu können. \newline
\Roman{7} . Drei feste Händler sollen beim Vergleich berücksichtigt werden. \newline
\Roman{8} . Der Warenkorb soll automatisch befüllt werden. \newline
\Roman{9} . Auf nicht verfügbare Einzelteile sollte hinreichend hingewiesen werden. \newline
\Roman{10} . Die Lieferkosten sollen beim Preisvergleich berücksichtigt werden. \newline 
\Roman{11} . Bei der Ausgabe des Vergleichs soll eine Verlinkung zum Produkt, sowie eine Liste der Bauteile, angezeigt werden. \newline
\subsubsection{Kannkriterien}
Test
\subsubsection{Wunschkriterien}
Einzelteile die besonders selten/teuer sollten gesondert aufgelistet werden \newline
Filterfunktion um zum Beispiel Figuren rauszufiltern \newline
Mehr Händler berücksichtigen \newline
Grafische Weboberfläche responsive designen \newline
Sticker berücksichtigen \newline
Benutzeraccount mit Wunschliste? (Historie) \newline
Eigene Bauanleitung auslesen lassen \newline
Aktuelle Angebote hervorheben \newline
Dashboard: Statistiken zum Suchverhalten unserer Benutzer / Wie erfolgreich war unser Programm? \newline

\subsubsection{Abgrenzungskriterien}
Unser Produkt wird keine Verkaufsplattform haben  \newline
Keine Bevorzugung von Händlern \newline
Wir berücksichtigen keine nicht offiziellen Klemmbausteine \newline
Website ist nur für deutschsprachige Benutzer \newline
Alte Sets ohne Stückliste in der Anleitung können nicht berücksichtigt werden (2006) \newline

\subsection{Produkteinsatz}
 Das Produkt soll Benutzer bei Kaufentscheidung unterstützen, außerdem unterstützt es die Benutzer Sets zu bauen die aus irgendwelchen Gründen nicht mehr verfügbar sind.\newline
Das Tool ist für Lego-Enthusiasten und Sammler gedacht, die nach dem günstigsten Angebot suchen. Das Tool kann auch von Einzelpersonen oder Unternehmen genutzt werden, die große Mengen an Lego-Sets kaufen möchten. \newline

\subsection{Produktfunktionen}
PDFs mit Crawler ansprechen können \newline
PDFs mittels OCR auslesen \newline
Einzelteile in Datenbank zwischenspeichern \newline
Informationen bei Händlern abfragen können \newline
Preisvergeich zwischen Sets und Einzelteilen \newline
Darstellung der Endergebnisse \newline

\subsection{Produktdaten}

\subsubsection{Bauteil}
Element-ID\newline
Name \newline

\subsubsection{Bauteil/Set-Preis}
Element-ID\newline
Preis\newline
Anbieter\newline
URL\newline

\subsubsection{Set}
Set-ID \newline
List: Element-ID \newline
Name \newline
Bool: Verfügbarkeit \newline
UVP \newline  \newline
\includegraphics[width=8cm]{pictures/chen1.png} \newline  \newline
\includegraphics[width=8cm]{pictures/klassendiagramm2.png}  \newline  \newline
\includegraphics[width=8cm]{pictures/klassendiagramm.png} 

\subsection{Produktleistungen}
Bestandsdatenbank soll aufgebaut werden mit Informationen aus den Anleitunge \newline
Preise sollen bei der Suche gecrawlt werden \newline
Die verwendete Datenbank muss in der Lage sein große Datenmengen zu speichern und zu verarbeiten \newline
Der Crawler soll in der Lage sein schnell Preisdaten bei den Anbietern zu crawlen, außerdem soll ein Algorithmus regelmäßig die neuen Anleitungen erkennen und dem Parser übergeben \newline

\subsection{Qualitätsanforderungen}
Dokumentation der Änderungshistorie \newline
Programmcode auskommentieren \newline

\subsection{Benutzungsoberfläche}
\includegraphics[width=8cm]{pictures/skizzeausgabe.jpg} \newline \newline
\includegraphics[width=8cm]{pictures/programmvorschau1.png}
\includegraphics[width=8cm]{pictures/programmvorschau2.png} \newline \newline
\includegraphics[width=8cm]{pictures/programmvorschau3.png}

\subsection{Nichtfunktionale Anforderungen}
Barrierefreiheit \newline
Intuitives Design \newline
ISO Normen aus dem MCIW-Modul einhalten \newline
Revisionsinstanzen Scrum etablieren \newline
Websitendaten schützen die gecrawlt werden \newline

\subsection{Technische Produktumgebung}
Datenbank \newline
Crawler \newline
Webserver \newline
OCR \newline
Angular \newline
HTML \newline
CSS \newline
Python \newline
Tex \newline
Git \newline

\subsection{Gliederung in Teilprodukte}
Bauanleitung suchen \newline
Bauanleitung parsen \newline
Datenbank füllen \newline
Preisvergleich \newline
Ausgabe generierern \newline
\includegraphics[width=8cm]{pictures/datenbankupdate.png} \newline  \newline
\includegraphics[width=8cm]{pictures/vorgaenge.png}


\subsection{Ergänzungen}
Aus rechtlichen Gründen sollen Anbieter vor dem Crawling angefragt werden \newline
Installationsanleitung und Handbuch sollen bereitgestellt werden \newline

\subsection{Globale Testfälle}
Set händisch recherchieren, um Soll-Wert für Unit-Test zu ermitteln und mit Algortihmusausgabe testen \newline
Set-ID mit Bauanleitung ohne Stückzahl \newline
Gülige Set-ID suchen \newline
Ungültige Set-ID suchen \newline
