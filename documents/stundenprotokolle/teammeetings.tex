\subsection{Stundenübersicht}

\begin{tabular}{|l|l|l|l|l|}
\hline
Termin & Datum & Zeitraum & Stunden & Erledigt \\
\hline
    Termin 1 & 19.04.2022 & 15:00 - 17:30 & 2,5 & Kickoff Termin + kurzer Austausch \\
	Termin 2 & 24.04.2022 & 15:30 - 17:00 & 1,5 & Besprechung zum Pflichtenheft \\
	Termin 3 & 27.04.2022 & 12:00 - 00:00 & 2,5 & Besprechung zum Pflichtenheft \\
\hline
\end{tabular}

\subsection{Protokolle}
\subsubsection{Protokoll zu Termin 1}
Protokollführer: Tim Turowski \newline
Zusätzliche Teilnehmer: Kolja Dunkel, Henning Ahlf \newline
Abwesend: keiner \newline \newline
Entwicklungsdeadline Mitte Dezember \newline
Erste Abgabe: Pflichtenheft nach Balzert \newline
Dafür haben wir 5 Wochen Zeit \newline
Termin vereinbart zum ersten Abgleich mit Kolja Dunkel: 2.Mai 10:00 Uhr \newline
Wir sollen Slag und Git verwenden \newline
Es können Entwicklungsumgebungen angefordert werden, Jira und Confluence \newline
Teamprotokoll muss geführt werden, weil wir etwa 330 Stunden leisten müssen \newline
Die Zeit wird für die Bewertung zur Relation zur Komplexität betrachtet \newline
Am Ende wird es Einzelnoten geben \newline
Wir müssen ausserdem einen Wochenplan aufstellen, in denen sollen wir Urlaube und Vorbereitungszeit für Prüfungen hinterlegen \newline
Es soll ausserdem ein Handbuch erstellt werden \newline
Wenn wir nachdem wir eine Technologien im Pflichtenheft dokumentiert haben, uns danach doch für eine Andere entscheiden, sollten wir das begründen 

\subsubsection{Protokoll zu Termin 2}
Protokollführer: Sven Wolf \newline
Zusätzliche Teilnehmer: - \newline
Abwesend: - \newline
Tim Turowski gibt Kurzeinweisung in LaTex und Git \newline
Hauptaufgabe des Treffens: füllen des Pflichtenhefts mit Inhalten \newline
Wir haben uns geeinigt Python nutzen zu wollen um unsere Kenntnisse in der Sprache zu verbessern \newline
Große Diskussionspunkte: \newline
- Eine Historie einfügen? Evtl. mit Benutzeraccounts? -> auf später verschoben (Wunschkriterium) \newline
- Wie wird die Ausgabe aussehen? Preise für alle Einzelteile? -> Extra Button "mehr Informationen anzeigen" \newline
- Wie wird die Datenbankstruktur aussehen? (technische Umsetzung) -> muss noch weiter geklärt werden \newline
Aufgaben/Fragestellungen fürs nächste Treffen: \newline
- Welche und Wieviele Shops vergleichen wir? \newline
- Wo und wie bekommen wir die Lego-Bauanleitungen? \newline
- Jeder fertigt eine Skizze zur Benutzeroberfläche an \newline
- Jeder fertigt ein Klassendiagramm an \newline
- Weitere Vorschläge für den Punkt 'Technische Produktumgebung' im Pflichtenheft finden \newline
Nächstes Treffen: Donnerstag, der 27.04.2023, 12:00Uhr Remote/Online mit dem Ziel das Pflichtenheft weiter zu füllen + oben genannten Aufgaben zu vergleichen \newline