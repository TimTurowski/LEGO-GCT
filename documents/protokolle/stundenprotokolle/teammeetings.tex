\subsection{Stundenübersicht}

\begin{flushleft}
		\begin{longtable}{p{2cm}p{2cm}p{3cm}p{2cm}p{8cm}}
        %% Tabellenkopf
            \toprule
            \textbf{Termin} & \textbf{Datum} & \textbf{Zeitraum} & \textbf{Erledigt} & \textbf{Erledigt}\\
            \midrule\endfirsthead
            \toprule
            \textbf{Termin} & \textbf{Datum} & \textbf{Zeitraum} & \textbf{Erledigt} & \textbf{Erledigt}\\
            \midrule\endhead
        %% Tabelleninhalte
            	Termin 1 & 19.04.2022 & 15:00 - 17:30 & 2,5 & Kickoff Termin + kurzer Austausch \\ \midrule
				Termin 2 & 24.04.2022 & 15:30 - 17:00 & 1,5 & Besprechung zum Pflichtenheft \\ \midrule
				Termin 3 & 27.04.2022 & 12:00 - 15:00 & 3 & Besprechung zum Pflichtenheft \\ \midrule
				Termin 4 & 02.05.2022 & 10:00 - 11:30 & 1,5 & Präsentation 1. Entwurf Pflichtenheft \\ \midrule
				Termin 5 & 04.05.2022 & 11:00 - 12:00 & 1 & Zwischenstandbesprechung \\
            \bottomrule
    \end{longtable}
\end{flushleft}

\section{Protokolle}
\subsection{Protokoll zu Termin 1}
Protokollführer: Tim Turowski \newline
Zusätzliche Teilnehmer: Kolja Dunkel, Henning Ahlf \newline
Abwesend: keiner \newline \newline
Entwicklungsdeadline Mitte Dezember \newline
Erste Abgabe: Pflichtenheft nach Balzert \newline
Dafür haben wir 5 Wochen Zeit \newline
Termin vereinbart zum ersten Abgleich mit Kolja Dunkel: 2.Mai 10:00 Uhr \newline
Wir sollen Slag und Git verwenden \newline
Es können Entwicklungsumgebungen angefordert werden, Jira und Confluence \newline
Teamprotokoll muss geführt werden, weil wir etwa 330 Stunden leisten müssen \newline
Die Zeit wird für die Bewertung zur Relation zur Komplexität betrachtet \newline
Am Ende wird es Einzelnoten geben \newline
Wir müssen ausserdem einen Wochenplan aufstellen, in denen sollen wir Urlaube und Vorbereitungszeit für Prüfungen hinterlegen \newline
Es soll ausserdem ein Handbuch erstellt werden \newline
Wenn wir nachdem wir eine Technologien im Pflichtenheft dokumentiert haben, uns danach doch für eine Andere entscheiden, sollten wir das begründen 

\subsection{Protokoll zu Termin 2}
Protokollführer: Sven Wolf \newline
Zusätzliche Teilnehmer: - \newline
Abwesend: - \newline \newline
Tim Turowski gibt Kurzeinweisung in LaTex und Git \newline
Hauptaufgabe des Treffens: füllen des Pflichtenhefts mit Inhalten \newline
Wir haben uns geeinigt Python nutzen zu wollen um unsere Kenntnisse in der Sprache zu verbessern \newline
Große Diskussionspunkte: \newline
- Eine Historie einfügen? Evtl. mit Benutzeraccounts? -> auf später verschoben (Wunschkriterium) \newline
- Wie wird die Ausgabe aussehen? Preise für alle Einzelteile? -> Extra Button "mehr Informationen anzeigen" \newline
- Wie wird die Datenbankstruktur aussehen? (technische Umsetzung) -> muss noch weiter geklärt werden \newline
Aufgaben/Fragestellungen fürs nächste Treffen: \newline
- Welche und Wieviele Shops vergleichen wir? \newline
- Wo und wie bekommen wir die Lego-Bauanleitungen? \newline
- Jeder fertigt eine Skizze zur Benutzeroberfläche an \newline
- Jeder fertigt ein Klassendiagramm an \newline
- Weitere Vorschläge für den Punkt 'Technische Produktumgebung' im Pflichtenheft finden \newline
Nächstes Treffen: Donnerstag, der 27.04.2023, 12:00Uhr Remote/Online mit dem Ziel das Pflichtenheft weiter zu füllen + oben genannten Aufgaben zu vergleichen

\subsection{Protokoll zu Termin 3}
Protokollführer: Dennis Behrendt \newline
Zusätzliche Teilnehmer: - \newline
Abwesend: - \newline \newline
Meeting per Discord und Github \newline
Weiterarbeit am Pflichtenheft: \newline
Wir haben über die "Produktdaten" diskutiert und diese in ein Klassendiagramm notiert \newline
Außerdem haben wir die "Produktleistungen" besprochenn und notiert \newline
 -Offenstehende Frage dazu: Wie groß wird das Datenaufkommen sein? \newline
Definierung der "Qualitätsanforderungen" \newline
Vergleich unserer Skizzen zu der "Benutzeroberfläche" und Ideensammlung \newline
 -Wie soll die Darstellung des Vergleichs (Preise, Teile, Händler) konkret aussehen? \newline
Definieren der "Nichtfunktionalen Anforderungen" \newline
Gemeinsame Recherche zu den Lego-Händlern (Welche Händler wählen wir genau? Bieten diese sowohl Set- sowie Einzelverkauf an? Lassen sich die Preise der Händler crawlen?) \newline
Fragestellung für die Projektgeber: \newline
 -Sollen wir für unsere Datenbank an Einzelteilen eine Vorlage verwenden oder diese automatisch per Crawler füllen? \newline
 
 \subsection{Protokoll zu Termin 4}
 Protokollführer: Tim Sibum \newline
Zusätzliche Teilnehmer: Kolja Dunkel, Henning Ahlf \newline
Abwesend: - \newline \newline
Präsentation der ersten Version des Pflichtenhefts für die Dozenten \newline\newline
Forderung der Dozenten:\newline
  - zulösende Problem mit in die Zielbestimmung aufnehmen\newline
  - Wie etwas funktionieren soll sollte in die Musskriterien aufgenommen werden
  \newline 
  - Priorisieren der Musskriterien\newline
  - Benutzerkonten System in Muss aufnehmen\newline
  - Mehr Händler berücksichtigen und responisve Desing in Kann aufnehemen\newline
  - Muss detaillierter beschreiben\newline
  - Automatische Warenkorb Befüllung in Wunsch aufnehmen\newline
  - Produktfunktionen nummerieren(Vorbedingung, Beschreibung\newline, 
  Nachbedingung, Möglicher Fehlerfall).\newline
  -Use Case Modellierung in Produktfunktion aufnehmen\newline
  - Benutzerprofile auch im Datensegment aufnehmen\newline
  - Benutzeroberfläche mit Workflow optisch ansprechender modellieren\newline
  - Alle Stichpunkte besser in ganzen Sätzen formulieren und W Fragen abdecken\newline 	  - Skizze zur Technischen Produktumgebung machen\newline
  - Installationsanleitung und Handbuch muss nicht unbedingt in das Pflichtenheft\newline 
  - Extradokument für Zeitennachweis, Protokolle und Pflichtenheft \newline 
Aufgaben für die Gruppe:\newline
  - Technische Anforderung spezifizieren was brauchen wir?(Datenbank)\newline        
  - Stundenkontigent auf 25 Wochen aufteilen und als Soll formulieren \newline
  - Projektplanung in GitHub organisieren\newline
  - Stichpunkte im Pflichtenheft ausformulieren \newline
  - erstellen von Diagrammen und Skizzen für das Pflichtenheft \newline

\subsection{Protokoll zu Termin 5}
Protokollführer: Hannes Scherer \newline
Zusätzliche Teilnehmer: - \newline
Abwesend: - \newline \newline
  - Deckblatt soll angefertigt werden \newline
  - Änderungshistorie soll optimiert werden z.B. durch auslagern des Pflichtenhefts. Zudem sollen Tabellen optimiert werden \newline
  - Dennis wollte gerne Römische Ziffern importieren \newline
  - Es wurden Überlegungen getätigt, ob die Funktion nach Händler filtern mitaufgenommen werden soll. \newline
  - Interface statt abstrakte Klasse in Tim S. Klassendiagramm \newline
  - Bestehende Grafiken sollen Optimiert werden \newline
  - Punkt 2.8 soll gestrichen werden \newline
  - In Punkt 2.9 erstellte Grafiken sollen überarbeitet werden \newline
  - Es wurde besprochen das wir uns damit befassen die Google Authentifizierung mit einzubauen \newline
  - Testfälle sollen erst nach den Produktfunktionen definiert werden \newline

