\subsection{Urlaubsplanung}

\begin{flushleft}
		\begin{longtable}{p{4cm}p{6cm}p{4cm}}
        %% Tabellenkopf
            \toprule
            \textbf{Projektmitglied} & \textbf{Zeitraum} & \textbf{Grund} \\
            \midrule\endfirsthead
            \toprule
            \textbf{Projektmitglied} & \textbf{Zeitraum} & \textbf{Grund} \\
            \midrule\endhead
        %% Tabelleninhalte
            	Tim Turowski & 25.05.2023 - 31.05.2022 & Urlaub \\ \midrule
				Team & 12.06.2023 - 07.07.2023 & Klausurenphase \\ \midrule
				Tim Turowski & 17.07.2023 - 24.07.2023 & Urlaub \\  \midrule
				Tim Turowski & 17.07.2023 - 24.07.2023 & Urlaub \\ \midrule
				Tim Sibum & 17.07.2023 - 24.07.2023 & Urlaub \\ \midrule
				Team & 18.09.2023 - 06.10.2023 & Klausurenphase \\ 
            \bottomrule
    \end{longtable}
\end{flushleft}

\section{Protokolle}
\subsection{Protokoll zu Termin 1}
Protokollführer: Tim Turowski \newline
Datum: 19.04.2023 \newline
Zusätzliche Teilnehmer: Kolja Dunkel, Henning Ahlf \newline
Abwesend: keiner \newline \newline
Entwicklungsdeadline Mitte Dezember \newline
Erste Abgabe: Pflichtenheft nach Balzert \newline
Dafür haben wir 5 Wochen Zeit \newline
Termin vereinbart zum ersten Abgleich mit Kolja Dunkel: 2.Mai 10:00 Uhr \newline
Wir sollen Slag und Git verwenden \newline
Es können Entwicklungsumgebungen angefordert werden, Jira und Confluence \newline
Teamprotokoll muss geführt werden, weil wir etwa 330 Stunden leisten müssen \newline
Die Zeit wird für die Bewertung zur Relation zur Komplexität betrachtet \newline
Am Ende wird es Einzelnoten geben \newline
Wir müssen ausserdem einen Wochenplan aufstellen, in denen sollen wir Urlaube und Vorbereitungszeit für Prüfungen hinterlegen \newline
Es soll ausserdem ein Handbuch erstellt werden \newline
Wenn wir nachdem wir eine Technologien im Pflichtenheft dokumentiert haben, uns danach doch für eine Andere entscheiden, sollten wir das begründen 

\subsection{Protokoll zu Termin 2}
Protokollführer: Sven Wolf \newline
Datum: 24.04.2023 \newline
Zusätzliche Teilnehmer: - \newline
Abwesend: - \newline \newline
Tim Turowski gibt Kurzeinweisung in LaTex und Git \newline
Hauptaufgabe des Treffens: füllen des Pflichtenhefts mit Inhalten \newline
Wir haben uns geeinigt Python nutzen zu wollen um unsere Kenntnisse in der Sprache zu verbessern \newline
Große Diskussionspunkte: \newline
- Eine Historie einfügen? Evtl. mit Benutzeraccounts? -> auf später verschoben (Wunschkriterium) \newline
- Wie wird die Ausgabe aussehen? Preise für alle Einzelteile? -> Extra Button "mehr Informationen anzeigen" \newline
- Wie wird die Datenbankstruktur aussehen? (technische Umsetzung) -> muss noch weiter geklärt werden \newline
Aufgaben/Fragestellungen fürs nächste Treffen: \newline
- Welche und Wieviele Shops vergleichen wir? \newline
- Wo und wie bekommen wir die Lego-Bauanleitungen? \newline
- Jeder fertigt eine Skizze zur Benutzeroberfläche an \newline
- Jeder fertigt ein Klassendiagramm an \newline
- Weitere Vorschläge für den Punkt 'Technische Produktumgebung' im Pflichtenheft finden \newline
Nächstes Treffen: Donnerstag, der 27.04.2023, 12:00Uhr Remote/Online mit dem Ziel das Pflichtenheft weiter zu füllen + oben genannten Aufgaben zu vergleichen

\subsection{Protokoll zu Termin 3}
Protokollführer: Dennis Behrendt \newline
Datum: 27.04.2023 \newline
Zusätzliche Teilnehmer: - \newline
Abwesend: - \newline \newline
Meeting per Discord und Github \newline
Weiterarbeit am Pflichtenheft: \newline
Wir haben über die "Produktdaten" diskutiert und diese in ein Klassendiagramm notiert \newline
Außerdem haben wir die "Produktleistungen" besprochenn und notiert \newline
 -Offenstehende Frage dazu: Wie groß wird das Datenaufkommen sein? \newline
Definierung der "Qualitätsanforderungen" \newline
Vergleich unserer Skizzen zu der "Benutzeroberfläche" und Ideensammlung \newline
 -Wie soll die Darstellung des Vergleichs (Preise, Teile, Händler) konkret aussehen? \newline
Definieren der "Nichtfunktionalen Anforderungen" \newline
Gemeinsame Recherche zu den Lego-Händlern (Welche Händler wählen wir genau? Bieten diese sowohl Set- sowie Einzelverkauf an? Lassen sich die Preise der Händler crawlen?) \newline
Fragestellung für die Projektgeber: \newline
 -Sollen wir für unsere Datenbank an Einzelteilen eine Vorlage verwenden oder diese automatisch per Crawler füllen? \newline
 
\subsection{Protokoll zu Termin 4}
Protokollführer: Tim Sibum \newline
Datum: 02.05.2023 \newline
Zusätzliche Teilnehmer: Kolja Dunkel, Henning Ahlf \newline
Abwesend: - \newline \newline
Präsentation der ersten Version des Pflichtenhefts für die Dozenten \newline\newline
Forderung der Dozenten:\newline
  - zulösende Problem mit in die Zielbestimmung aufnehmen\newline
  - Wie etwas funktionieren soll sollte in die Musskriterien aufgenommen werden
  \newline 
  - Priorisieren der Musskriterien\newline
  - Benutzerkonten System in Muss aufnehmen\newline
  - Mehr Händler berücksichtigen und responisve Desing in Kann aufnehemen\newline
  - Muss detaillierter beschreiben\newline
  - Automatische Warenkorb Befüllung in Wunsch aufnehmen\newline
  - Produktfunktionen nummerieren(Vorbedingung, Beschreibung\newline, 
  Nachbedingung, Möglicher Fehlerfall).\newline
  -Use Case Modellierung in Produktfunktion aufnehmen\newline
  - Benutzerprofile auch im Datensegment aufnehmen\newline
  - Benutzeroberfläche mit Workflow optisch ansprechender modellieren\newline
  - Alle Stichpunkte besser in ganzen Sätzen formulieren und W Fragen abdecken\newline 	  - Skizze zur Technischen Produktumgebung machen\newline
  - Installationsanleitung und Handbuch muss nicht unbedingt in das Pflichtenheft\newline 
  - Extradokument für Zeitennachweis, Protokolle und Pflichtenheft \newline 
Aufgaben für die Gruppe:\newline
  - Technische Anforderung spezifizieren was brauchen wir?(Datenbank)\newline        
  - Stundenkontigent auf 25 Wochen aufteilen und als Soll formulieren \newline
  - Projektplanung in GitHub organisieren\newline
  - Stichpunkte im Pflichtenheft ausformulieren \newline
  - erstellen von Diagrammen und Skizzen für das Pflichtenheft \newline

\subsection{Protokoll zu Termin 5}
Protokollführer: Hannes Scherer \newline
Datum: 04.05.2023 \newline
Zusätzliche Teilnehmer: - \newline
Abwesend: - \newline \newline
  - Deckblatt soll angefertigt werden \newline
  - Änderungshistorie soll optimiert werden z.B. durch auslagern des Pflichtenhefts. Zudem sollen Tabellen optimiert werden \newline
  - Dennis wollte gerne Römische Ziffern importieren \newline
  - Es wurden Überlegungen getätigt, ob die Funktion nach Händler filtern mitaufgenommen werden soll. \newline
  - Interface statt abstrakte Klasse in Tim S. Klassendiagramm \newline
  - Bestehende Grafiken sollen Optimiert werden \newline
  - Punkt 2.8 soll gestrichen werden \newline
  - In Punkt 2.9 erstellte Grafiken sollen überarbeitet werden \newline
  - Es wurde besprochen das wir uns damit befassen die Google Authentifizierung mit einzubauen \newline
  - Testfälle sollen erst nach den Produktfunktionen definiert werden \newline

\newpage

\subsection{Protokoll zu Termin 6}
Protokollführer: Tim Turowski \newline
Datum: 08.05.2023 \newline
Zusätzliche Teilnehmer: - \newline
Abwesend: - \newline \newline
Aus der heutigen Besprechung haben sich folgende Arbeitspakete ergeben, die bis zur nächsten Woche noch abgearbeitet sein müssen:
\begin{flushleft}
	\begin{longtable}{p{3cm}p{2cm}p{2cm}p{6cm}}
		%% Tabellenkopf
		\toprule
		\textbf{Arbeitspaket} & \textbf{Kapitel} & \textbf{Bearbeiter} & \textbf{TODO}\\
		\midrule\endfirsthead
		\toprule
		\textbf{Arbeitspaket} & \textbf{Kapitel} & \textbf{Bearbeiter} & \textbf{TODO}\\
\midrule\endfirsthead
		%% Tabelleninhalte
		AP1 & Kapitel 2 &  Tim T & Mit in die Abgrenzungskriterien aufnehmen, dass keine Retailpreise berücksichtigt werden sollen, wir nehmen nur ofizielle Bauanleitungen von der Legodatenbank \\ \midrule
		AP2& Kapitel 4 &  Hannes & Funktionen formulieren für: Suchmaske, Anzeigenhistorie, Registrierung, Anmeldung, Stückliste, Minimieren Stückliste, Historie Wiederholen, Neue Legosets crawlen, Neue PDFs auslesen, PDF Informationen auslesen, Informationen in Datenbank ablegen, Preise bei Händlern crawlen, Einzelteile die neu sind in DB ablegen, Preise sollen automatisiert geupdatet werden. Außerdem Warenkorbfunktion nicht mehr berücksichtigen und alles in logische Reihenfolge bringen\\ \midrule
		AP3 & Kapitel 5.1 &  Tim S & Es fehlen Informationen zur Datenverwaltung, rechte Seite muss überarbeitet werden, außerdem noch Nutzer hinzufügen. Einzelangebot muss Marktpreis werden und Verbindung zum Set bekommen \\ \midrule
		AP4 & Kapitel 8 & Caner & Folgende Teilproduktverbindungen müssen noch deutlicher werden: Phyton -> SQL Datenbank, Crawler -> PDF -> OCR (Parser) -> Python, Crawler -> Website -> Phyton  \\ \midrule
		AP5 & Kapitel 8 &  Team &  Anforderungen definieren für VM, Websever, DB, Fileserver \\ \midrule
		AP6 & - &  Sven &  Use Case Diagramm erstellen \\ \midrule
		AP7 & - & Tim S &  Klassendiagramm überarbeiten, es gibt eine Liste die PreisId hat und zu alle Preise verwaltet, die gefunden wurden. Weitere Attribute könnten sein: TeileID, Preis, Händler...\\ \midrule
		AP8 & Kapitel 9 &  & Strukturierung Ändern: 9.1 Letzten Satz streichen, 9.2 Teilefilter, 9.3 In Anleitungscrawler umändern, 9.4 muss auch geändert werden, 9.6 in Geschäftslogik ändern \\ \midrule
		AP8 & Kapitel 10 & Dennis &  T60 streichen, ansonsten noch weitere hinzufügen, dabei mit Hannes absprechen und an AP2 orientieren \\ 
		\bottomrule
	\end{longtable}
\end{flushleft}
In naher Zukunft müssen wir unsere Stundenkontigente außerdem noch auf die 25 Wochen aufteilen. 
\newpage
\subsection{Protokoll zu Termin 7}
Protokollführer: Tim Turowski \newline
Datum: 08.05.2023 \newline
Zusätzliche Teilnehmer: - \newline
Abwesend: - \newline \newline
Wir haben uns das Pflichtenheft nochmal genauer angeguckt und besprochen wie weit jeder bei seinen Aufgaben ist, die sich aus dem letzten Termin ergeben haben. \newline
Wir haben eine bessere Strukturierung für die Produktfunktionen festgelegt \newline
Außerdem haben wir besprochen in wie fern noch Bilder eingefügt werden müssen, die eigentlich schon im Repository liegen \newline
Wir haben das weitere Vorgehen besprochen und festgelegt, dass wir die weitere Projektplanung mit Github machen wollen \newline
Zum besseren Entwickeln im Team möchten wir mit einem Kanbanboard nutzen \newline
\newpage
\subsection{Protokoll zu Termin 8}
Protokollführer: Dennis Behrendt \newline
Datum: 08.05.2023 \newline
Zusätzliche Teilnehmer: Kolja Dunkel, Prof. Henning Ahlf \newline
Abwesend: Hannes Scherer \newline \newline
Wir hatten unser Meeting mit Prof. Ahlf und Herr Dunkel. Das Pflichtenheft wurde angenommen, es kann also mit den Vorbereitungen für die Implementationsphase begonnen werden. \newline
Einige kleine Punkte im Plichtenheft müssen jedoch noch am Plichtenheft verbessert und Prof. Ahlf geschickt werden: \newline
  - "Eigene PDF's laden" zu den Wunschkriterien \newline
  - Administratorfunktionen zu den Produktfuntionen hinzufügen \newline
  - Use-Case-Diagramm zu den Administratorfunktionen einfügen \newline
  - Set und Einzelteile an die Anbieter binden (Produktdaten) \newline
  - 32GB Arbeitsspeicher zu viel, Server Linux-basiert wählen (Ubuntu CentOS), 1TB zuviel \newline
  - Händler, die wir auswählen, im notieren \newline
Daraufhin hatten wir ein einstündiges Discord-Meeting, indem wir unser künftiges Vorgehen besprochen haben: \newline
  - Abstraktion der Aufgabenbereiche \newline
  - Erstellen eines Projektboards \newline
  - Aufteilung der Aufwandszeit der Teilaufgaben zu der Gesamtzeit des Projekts \newline
Daraus haben sich ebenfalls folgende Rechearbeiten ergeben, die bis zur nächsten Woche noch abgearbeitet sein müssen:
\begin{flushleft}
	\begin{longtable}{p{4cm}p{7cm}p{3cm}}
		%% Tabellenkopf
		\toprule
		\textbf{Arbeitspaket} & \textbf{Thema} & \textbf{Bearbeiter}\\
		\midrule\endfirsthead
		\toprule
		\textbf{Arbeitspaket} & \textbf{Thema} & \textbf{Bearbeiter}\\
\midrule\endfirsthead
		%% Tabelleninhalte
		AP1 & Crawler &  Tim S, Sven\\ \midrule
		AP2& Parser PDF &  Dennis, Caner\\ \midrule
		AP3 & Benutzeroberfläche & Tim T, Hannes\\ \midrule
		AP4 & Datenbanken & Caner, Tim S\\ \midrule
		AP5 & Python und Geschäftslogik &  Alle\\ \midrule
		AP6 & Accountverwaltung &  Sven, Tim S, Dennis\\ \midrule
		AP7 & Webserver & Hannes, Tim T\\ \midrule
		AP8 & Kanban und Sorum & Tim T, Hannes\\
		\bottomrule
	\end{longtable}
\end{flushleft}
\newpage
\subsection{Protokoll zu Termin 9}
Protokollführer: Tim Turowski\newline
Datum: 08.05.2023 \newline
Zusätzliche Teilnehmer:  \newline
Abwesend: \newline \newline
Wir mussten das Meeting Remote durchführen, da 2 Beteiligte sonst krank ausgefallen wären.\newline
Zu Beginn haben TIm S und Tim T ihre Recherchen zu Crawlern und Parsern vorgestellt \newline
Danach gab es eine kurze Diskussion drüber was genau beachtet werden muss.\newline
Von Caner wurden 2 Parser vorgestellt: 1 Python-parser, 1 KI-Parser, wir wollen es mit dem Python-Parser versuchen um nicht noch mehr Schnittstellen aufzumachen.\newline
Zu Angular gab es einen kleinen Input von Hannes. Technik: Angular\newline
Caner hat kurz was zur Geschäftslogik gesagt und wie wir die Acc einpflegen könnten \newline
Danach haben wir uns die Technik SQL-Alchemy angeguckt und eine Diskussion über Datenbanken geführt \newline
Hannes hat schon einen Webserver aufgesetzt. Apache + DB läuft auch bereits. Man kann wenn wir eine Serveranbindung bekommen, das bestehende System auf den neuen Ubutu-Server konen, sofern es die selbe Version ist, die wir im Pflichenheft eingetragen haben\newline
Hannes hatte noch eine Powerpoint über Kanban und Scrum gehalten, die sehr interessant war. Für die Umsetzung haben wir einen Jirazugriff bei Herrn Dunkel angefragt\newline
\newpage
\subsection{Protokoll zu Termin 10}
Protokollführer: Tim Turowski\newline
Datum: 17.07.2023 \newline
Zusätzliche Teilnehmer:  \newline
Abwesend: \newline \newline
Gestartet wurde das Meeting mit einer Präsentierrunde der bisherigen Fortschritte. Hierzu zählen der Preisecrawler, welcher funktioniert aber noch zeitlich optimiert werden muss. Außerdem steht eine Grundidee für den PDF-Crawler. Problem welches hierbei aufgetreten ist, ist das die Lego.com Seite mit JavaScript läuft und unser bisher eingesetzter Crawler damit nicht zurecht kommt. Als Workaround haben wir die Lösung über die Seite Steinelager.com auf die PDFs der offiziellen Lego.com Seite zuzugreifen. Der OCR Parser ist ebenfalls beinhahe Einsatzbereit. Diskutiert wurden ebenfalls die Folgenden Themen: \newline \newline
benötigen wir eine Löschroutine für die PDFs? -> Ja \newline
ist unser bisheriges Datenbankschema noch aktuell? -> Nein. Kleine Anpassungen in den Attributen sind notwendig \newline
eine Arbeitsroutine für die Aktualisierung -> wie Aktualisierung? Keine Lösung gefunden, auf weiteres Meeting verschoben 
\newpage
\subsection{Protokoll zu Termin 11}
Protokollführer: Sven Wolf\newline
Datum: 24.07.2023 \newline
Zusätzliche Teilnehmer:  \newline
Abwesend: \newline \newline
In dem Termin haben wir gemeinsam die einzelnen Komponenten zusammengefügt und eine kleine Geschäftslogik gebaut, an sich hatte jeder seinen eigenen Programmierstil, was am Anfang etwas schwierig gewesen ist, die einzelnen Methoden sind aber so definiert gewesen, dass man die Teile problemlos zusammenfügen konnten. Der erste Test als geschlossene Kette hat an sich funktioniert, es sind jedoch keine Regeln im Programm, welche Ausnahmen berücksichtigen. Am Ende haben wir diese Regeln definiert und aufgeschrieben. Außerdem hat Hannes seinen Stand zu Angular mittgeteilt und wollte weitere Pakete und unser Repository via Git auf den Server "klonen". 
\newpage
\subsection{Protokoll zu Termin 12}
Protokollführer: Tim Turowski\newline
Datum: 31.07.2023 \newline
Zusätzliche Teilnehmer:  \newline
Abwesend: \newline \newline
Wir haben gesagt dass wir die Dokumentationen bearbeiten müssen. Dabei müssen wir aufschreiben, was wir anders gemacht haben als geplant, das sind nach der ersten Sammlung: SQL-Abfrage von Frontend auf DB, DB-Struktur und dass PDFs statt von Lego.com jetzt von Steinelager gezogen werden (Workaround). Dann müssen Weeklyprotokolle überarbeitet werden bis dahin unter anderem auch das was ich grade hier schreibe. Im Meeting mit Prof. Ahlf und Herr Dunkel wollen wir auf folgende Themen eingehen: Plan zeigen, Jira-Kanbanboard (wie weit sind wir vor Plan?) Crawler und Paser haben schneller geklappt als gedacht. Welche Daten können wir von zu Hause aus füllen? Welche Daten muss der Server in Routinen in Db updaten Caner wollte sich mit Strukturen der Benutzerdatenbank auseinandersetzen Für alle einmal die Hausaufgabe die Stunden zu aktualisieren, sich schonmal mit Angular auseinandersetzen Des weiteren müssen wir uns langsam Gedanken machen wie die Logik aussehen muss für die aktualisierung der Daten in der Datenbank: Was ist wenn ein Teil vorgemerkt ist, aber noch keine Bauanleitung hat? TODO: Name und ID müssen von LegoSets übergeben werden PDFs müssen automatisch nach crawlen gelöscht werden CSV Reader
\newpage
\subsection{Protokoll zu Termin 13}
Protokollführer: Hannes Scherer \newline
Datum: 07.08.2023 \newline
Zusätzliche Teilnehmer: Kolja Dunkel, Henning Ahlf  \newline
Abwesend: Caner Kaya \newline \newline
Tim Turowski eröffnet das Meeting und erklärt die Arbeitspakete und die geleisteten Stunden mit Graphen. 
Tim Turowski erzählt woran die Letzten Wochen gearbeitet wurden.
Tim Turowski erzählt was in den Letzten Meetings so besprochen wurde.
Sven Wolf beginnt mit der Präsentation welche Arbeitspakete bis jetzt Abgeschlossen wurden
Sven Wolf erklärt an welchen Arbeitspaketen aktuell gearbeitet wird. 
Tim Sibum zeigt ein wenig Praktisches und erklärt den Downloader und den Parser und geht zudem auf den Crawler ein
Tim Sibum fragt nach Fragen
Herr Dunkel stellt die Frage nach den Preisen da aus der Präsentation hervorging das dieses Arbeitspaket aktuell noch entwickelt wird: „Wie werden die Preise aktualisiert“ (sinngemäße Wiedergabe) 
Tim Sibum und Hannes Scherer antworten das wir aktuell mehrere Lösungen diskutieren und wir glauben das Cronjobs auf dem Server eine gute Lösung sein könnte.
Die Zeitentwicklung wird als Positiv erachtet der Entwicklungstand für fortgeschritten erklärt „auf einem guten Weg“(sinngemäße Wiedergabe)
Das Projektteam bedankt Sich für die Aufmerksamkeit und verabschiedet sich
\subsection{Protokoll zu Termin 15}
Protokollführer: Tim Sibum \newline
Datum: 12.10.2023 \newline
Zusätzliche Teilnehmer:\newline
Abwesend:Dennis Behrendt, Tim Turowski \newline \newline
Besprechen des Aktuellenstandes:\newline
- Hannes hat den aktuellen Stand des Frontendpräsentiert\newline
- Tim S hat den Bricklink Crawler und den Element-Id Übersetzer gezeigt\newline
Ziele für die nächsten Wochen:\newline
- Datenbankanbindung im Frontend\newline
- Routinen für den Server\newline
\newpage
\subsection{Protokoll zu Termin 16}
Protokollführer: Dennis Behrendt \newline
Datum: 24.10.2023 \newline
Zusätzliche Teilnehmer:\newline
Abwesend:Hannes Scherer \newline \newline
API für DatenBank Entities erstellt \newline
WAs können wir bisher:\newline
- Preise crawlen aus Brickling-Shops\newline
- Brickling Set-Ids in offizielle Lego-Ids transformieren\newline
- Neuer Crawler: agiert wie Bot und klickt sich durch Seiten \newline
- Watchlist: Holt alle Sets aus Jahr x; checlt mit Logs aus Sets; ob diese vorhanden sind; mit time-to-live zeitlich bestimmen, wenn diese gesucht werden
- Discord Bot: Gibt Meldungen, wenn ein Set hinzugefügt wurde
Was haben wir vor?\newline
- TimS: Python-Skripte für aktuelle Preise, Server Unterstützung \newline
- Sven: Server Backend einrichten, Angular in Server einstellen \newline
- Caner: Angular lernen um zu helfen \newline
- Dennis: Impressum, Menü-Leiste, Frontend \newline
- TimT: Benuteraccount, Frontend \newline
- Hannes: Suche, Frontend \newline
- Routinen für den Server\newline


