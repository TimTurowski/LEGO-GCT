\subsubsection{Geleistete Stunden}
\begin{flushleft}
		\begin{longtable}{p{2cm}p{2,5cm}p{2cm}p{10,5cm}}
        %% Tabellenkopf
            \toprule
            \textbf{Termin} & \textbf{Datum} & \textbf{Zeitraum} & \textbf{Erledigt}\\
            \midrule\endfirsthead
            \toprule
            \textbf{Termin} & \textbf{Datum} & \textbf{Zeitraum} & \textbf{Erledigt}\\
            \midrule\endhead
        %% Tabelleninhalte
            	20.04.2023 & 08:30-10:00 & 1,5 & Softwareideen aufschreiben\\ \midrule
    			20.04.2023 & 12:00-12:30 & 0,5 & Softwareideen aufschreiben\\ \midrule
    			23.04.2023 & 11:00-11:45 & 0,75 & Testen von Crawlern(Java)\\ \midrule
    			24.04.2023 & 11:30-12:30 & 1 & Testen von Crawlern(Java)\\ \midrule
    			25.04.2023 & 08:45-09:30 & 0,75 & Repository Klonen und mit Latex vertraut machen\\ \midrule
    			26.04.2023 & 14:15-16:00 & 1,75 & ERM und UML Diagramm zur Datenstruktur\\ \midrule
    			27.04.2023 & 08:45-10:45 & 2 & Sequenzdiagramm, Zustandsdiagramm und GUI Skizze\\
    			02.05.2023 & 14:30-16:00 & 1,5 & UML Diagramm Benutzerdaten\\ \midrule
    			04.05.2023 & 9:30-11:00 & 1,5 & UML Diagramm Architektur\\ \midrule
    			04.05.2023 & 16:00-16:30& 0,5 & Datenstrukturen beschreiben\\ \midrule
    			06.05.2023 & 10:00-11:45 & 1,75 & ausarbeiten der Datenstrukturen UML + Beschreibung \\ \midrule
    			07.05.2023 & 9:10-10:20 & 1,16 & Sequenzdiagramm Einzelteile\\ \midrule
    			08.05.2023 & 10:00-10:40 & 0,66, & Recherche Nutzersystem\\ \midrule
    			10.05.2023 & 13:00-13:45 & 0,75 & Sequenzdiagramm überarbeiten\\ \midrule
    			11.05.2023 & 8:00-10:00 & 2 & alle UML Diagramme fertigstellen und verbessern + Datenstruktur Beschreibung fertigstellen\\ \midrule

            \bottomrule
    \end{longtable}
\end{flushleft}